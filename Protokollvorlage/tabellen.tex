
im Gegensatz zu Bildern haben Tabellen \textbf{grundsätzlich} Überschriften!
\begin{table}[b]
 %Auch hier unterschiedliche Texte für Überschrift und Verzeichnis:
  \caption[Die erste Tabelle hier]{Eine Tabelle mit linksb\"undiger, 
zentrierter und rechtsb\"undiger Spalte. Sie steht wegen des $[b]$ (von bottom) unten auf der Seite.
%Sie steht als erstes auf dieser 
%Seite, obwohl sie im Quelltext nach \autoref{fig:sinus} und \autoref{fig:sinus2} steht.
}
  \centering
  \begin{tabular}{|l||c|r|}
  \hline
    A & B & $\Gamma$\\
    \hline
    \hline
    \hline
    ddd & eee & $\pi$\\
    \hline
    $\Pi$ & h & iiii\\
  \hline 
  \end{tabular}\label{tab:table2}
\end{table}
\begin{table}[t]
\caption[Zwei Tabellen nebeneinander]{Hier sind 2 Tabellen mit Minipages nebeneinander gesetzt. Wie die Minipages intern/extern stehen kann man 
setzen. Die Breiten kann man per Hand setzten oder automatisch.\\ In der linken Tabelle sind Zahlen sehr gut vergleichbar sind. Der Befehl \mbox{@\{\}} unterdrückt hier Zwischenräume. Diese Tabelle ist zentriert.\\
In der rechten Tabelle ist eine Zeilenhöhe künstlich, durch einen unsichtbaren Maßstab verändert. Hier sind in einer Zeile auch Spalten zusammengefasst.}
\begin{minipage}{3cm}
% die Breite der Minipage muss spezifiziert werden
\centering
\begin{tabular}{r@{}c@{}l@{}}
\hline 100 & , & 1 \\ 
\hline 10 & , & 01 \\ 
\hline 1 & , & 001 \\ 
\hline 
\end{tabular} 
\end{minipage}
% diese Befehler erlauben horizontales schieben
% \hfill
\hspace*{3cm}
\begin{minipage}[t][3cm][r]{0.5\textwidth}
% in den eckigen klammern stehen optional 
% äußere Position, Höhe, innere Position
%\flushleft/right erlaubt auch Positionsbestimmung
\begin{tabular}{|l|l|l|l|r|}
\hline
%rule{Breite}{Höhe} setzt einen nicht sichtbaren Maßstab (Breite 0) ein
\rule{0cm}{3cm}  1& 2& 3& 4& 5 \\\hline
  Spalte 1 & \multicolumn{3}{|c|}{|Spalte 2--4} & Spalte 5\\
 \hline
\end{tabular}
\end{minipage}\label{tab:table}
\end{table}