\section{Formeln}\label{sec:formeln}
Für Formeln sollte man unbedingt \AmS\TeX\  verwenden, das ist einfach zu bedienen und liefert bessere Ergebnisse als Standard-\LaTeX.

Dies macht der Befehl \textsf{align}.
% Achtung hier dürfen keine Lehrzeilen eingeführt werden !
\begin{align}
 \int_V\rho\,dV&=\sum_{i=1}^n m_i \quad=\quad M  \\
\intertext{Tiefer stellt man durch ein vorgestelltes underline (\_), höher durch Circumflex \^{}. Mehrere 
Buchstaben muss man in eine geschweifte Klammer einbinden.}
  S &= k\cdot\log{W} \\
\intertext{Vor Funktionen steht immer ein Backslash, z.B. \textsf{\textbackslash exp, \textbackslash sin}:}  
 c^2&=a^2+b^2 \nonumber \\
\intertext{Oben ist mit \textsf{\textbackslash nonumber} die Numerierung unterdrückt. }
 \chi&=5 \\
  x &= y &= z \label{zeile4}\\
  z &= a &= 1 \label{zeile5}
\end{align}
In Zeile \ref{zeile4} und \ref{zeile5} stehen mehrere Gleichheitszeichen übereinander.
Das geht schöner mit einem \textsf{alignat*} ohne Gleichngsnummern (macht der *).
\begin{xalignat*}{2}
 A &= B   &\qquad  B &= C \\
 C &= D   &\qquad  D &= E 
\end{xalignat*}
%\quad und \qquad machen Abstände
Die folgenden Gleichungen sind mit  \textsf{gather} zentriert untereinander geschrieben:
\begin{gather}
 A = B   \\ 
 A+B  = D
\end{gather}
Hiermit kann man mehrere Gleichungen übereinander mit nur einer Gleichungsnummer schreiben
\begin{equation}
  \begin{split}
 A &= B   \\ 
   &= T
  \end{split}\label{eq:equation}
\end{equation}
Wenn man mal eine wirklich lange Gleichung hat benutzt man \textsf{multline}.
\begin{multline}
\text{Dies ist eine}\\
\text{wirklich\qquad\qquad}\\
\text{unheimlich}\\
\text{\qquad\qquad l\"angliche}\\
\text{Gleichung}
\end{multline}

